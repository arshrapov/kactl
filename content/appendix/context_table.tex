\documentclass{article}
\usepackage{array} % Для улучшенного управления столбцами
\usepackage{graphicx} % Для поворота таблицы

\setlength{\arrayrulewidth}{0.25mm}  % Увеличиваем толщину линий
\setlength{\tabcolsep}{9pt} % Увеличиваем отступы между столбцами
\renewcommand{\arraystretch}{2} % Увеличиваем высоту строк

\begin{document}

    \begin{table}[h!]
%        \rotatebox{90}
        \centering
        \begin{tabular}{|c|c|c|c|c|c|c|c|c|c|c|c|c|}
            \hline
            \textbf{Condition} & A & B & C & D & E & F & G & H & I & J & K & L \\
            \hline
            Rafa read &   &   &   &   &   &   &   &   &   &   &   &   \\
            \hline
            Rafa has sol &   &   &   &   &   &   &   &   &   &   &   &   \\
            \hline
            Rafa in process &   &   &   &   &   &   &   &   &   &   &   &   \\
            \hline
            Sard read &   &   &   &   &   &   &   &   &   &   &   &   \\
            \hline
            Sard has sol &   &   &   &   &   &   &   &   &   &   &   &   \\
            \hline
            Svat read &   &   &   &   &   &   &   &   &   &   &   &   \\
            \hline
            Svat has sol &   &   &   &   &   &   &   &   &   &   &   &   \\
            \hline
            Svat in process &   &   &   &   &   &   &   &   &   &   &   &   \\
            \hline
        \end{tabular}
    \end{table}

\end{document}
